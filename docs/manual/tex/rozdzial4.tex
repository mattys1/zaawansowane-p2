	\newpage
\section{Implementacja}		%4
%Opisać implementacje algorytmu/programu. Pokazać ciekawe fragmenty kodu
%Opisać powstałe wyniki (algorytmu/nrzędzia)

\subsection{Ogólne informacje o implementacji klas}

\subsubsection{Klasa BSTTree}

Drzewo jest zaimplementowane jako jeden plik \texttt{.hpp}. Nie jest podzielone na plik implementacji oraz nagłówek, ponieważ jest ono szablonem. Deklaracja klasy oraz prywatne elementy wyglądają następująco:

\begin{lstlisting}[caption=Deklaracja drzewa BST, label={lst:BSTprivate}, language=C++]
	template <typename T>
	class BSTTree {
		private:
		struct Tree {
			T contents;
			Tree* parent;
			Tree* left;
			Tree* right;
			
			Tree(T _contents, Tree* _parent = nullptr, Tree* _left = nullptr, Tree* _right = nullptr):
			contents { _contents },
			parent { _parent },
			left { _left },
			right { _right } {}
			
			~Tree() {
				delete left;
				delete right;
			}
		};
		
		Tree* root;
		
		void recursive_add(const T& element, Tree*& node, Tree* parentNode = nullptr) {
			if(node == nullptr) {
				node = new Tree(element);
				
				if(parentNode == nullptr) {
					return;
				}
				
				node->parent = parentNode;
				
				if(node->contents >= parentNode->contents) {
					parentNode->right = node;
				} else {
					parentNode->left = node;
				}
				
				return;
			}
			
			if(element >= node->contents) {
				recursive_add(element, node->right, node);
			} else {
				recursive_add(element, node->left, node);
			}	
		}
		
		void preorder_traverse_recursive(Tree* node, std::vector<Tree*>& traversedTrees) {
			if(node == nullptr) {
				return;
			}
			
			traversedTrees.push_back(node);
			
			preorder_traverse_recursive(node->left, traversedTrees);
			preorder_traverse_recursive(node->right, traversedTrees);
		}
		
		void inorder_traverse_recursive(Tree* node, std::vector<Tree*>& traversedTrees) {
			if(node == nullptr) {
				return;
			}
			
			inorder_traverse_recursive(node->left, traversedTrees);
			
			traversedTrees.push_back(node);
			
			inorder_traverse_recursive(node->right, traversedTrees);
		}
		
		void postorder_traverse_recursive(Tree* node, std::vector<Tree*>& traversedTrees) {
			if(node == nullptr) {
				return;
			}
			
			postorder_traverse_recursive(node->left, traversedTrees);
			postorder_traverse_recursive(node->right, traversedTrees);
			
			traversedTrees.push_back(node);
		}
		public:
			...
\end{lstlisting}

Jak widać w kodzie zamieszczonym na listingu nr. \ref{lst:BSTprivate}., Klasa jest wrapperem dla structa \texttt{Tree}, zadeklarowanego na linijce nr. 4. Struct ten ma trzy wskaźniki - \texttt{left} dla elementu lewego, \texttt{right} dla elementu prawego, \texttt{parent} dla elementu będącego rodzicem oraz zawartość \texttt{contents}, opsującą element zawarty w danym węźle. 

W linii 10 zadeklarowany jest konstruktor, który domyślnie ustawia wartości wskaźników rodzica, dziecka lewego oraz prawego na \texttt{nullptr}. Istnieje możliwość sprecyzowania wartości wskaźników.

Od linii 16 do 18 znajduje się destruktor drzewa. W destruktorze węzeł lewy i prawy jest usuwany.

Od linijki nr. 24, do końca fragmentu, zawarty jest szereg pomocniczych metod prywatnych. Ich działanie zostanie omówione przy dyskusji korzystających z nich metod publicznych.

Jedną z tych metod publicznych jest \texttt{add()}, którą widać na listingu nr. \ref{lst:BSTadd}

\begin{lstlisting}[caption=Metoda \texttt{add()}, label={lst:BSTadd}, language=C++]
	void add(const T& element) {
		recursive_add(element, root);
	}

\end{lstlisting}

Metoda ta przyjmuje parametr \texttt{element}, który będzie wartością nowo dodanego węzła. W swoim ciele, wywołuje ona prywatną metodę \texttt{recursive\_add()} z parametrami \texttt{element} i korzeniem drzewa \texttt{root}. Jej definicja jest zawarta w linijce 24, listingu nr. \ref{lst:BSTprivate}. 

Metoda \texttt{recursive\_add()} działa na zasadzie rekurencji. Na początku funkcji sprawdzane jest czy teraźniejszy \texttt{node} to \texttt{nullptr}. Jeżeli tak, oznacza to, że doszło się do końca drzewa i można ten wskaźnik ustawić jako nowy węzeł. Na linijce nr. 28, sprawdzane jest czy rodzic wskaźnika \texttt{node} to \texttt{nullptr}. Jeżeli tak, oznaczałoby to, że pracujemy ze wskaźnikiem \texttt{root}. Wracamy wtedy wcześnie z funkcji, dlatego że \texttt{root} nie ma rodzica, więc nie chcemy manipulować wskaźnikami tego rodzica. Następnie, po zakończeniu bloku \texttt{if}, ustawiane są wskaźniki rodzica. Na linijce nr. 43, metoda wywoływana jest ponownie, w zależności od tego czy zawartość \texttt{node} jest większa od liczby która ma być dodana, czy nie.

Kolejną metodą na jaką można zwrócić uwagę to \texttt{traverse\_preorder()}, zawarta na listingu nr. \ref{lst:BSTtraverse_preorder}. 

\begin{lstlisting}[caption=Metoda \texttt{add()}, label={lst:BSTtraverse_preorder}, language=C++]
	std::vector<T> traverse_preorder(void) {
		std::vector<Tree*> traversedTrees;
	
		preorder_traverse_recursive(root, traversedTrees);

		return traversedTrees 
		| std::ranges::views::transform(
				[](const Tree* tree) { return tree->contents; }
		) 
		| std::ranges::to<std::vector>(); 
	}

\end{lstlisting}

Metoda \texttt{traverse\_preorder()}, ma za zadanie zwrócić wektor wartości węzłów ułożonych w kolejności preorder drzewa. Ku temu celu, na początku metody, w linijce nr. 2 listingu nr. \ref{lst:BSTtraverse_preorder}, deklarowany jest wektor \texttt{traversedTrees}, który będzie wypełniony wskaźnikami \texttt{Tree*} w kolejności preorder. Do wypełnienia tego wektora, używana jest prywatna metoda \texttt{preorder\_traverse\_recursive()} z parametrami \texttt{root} i \texttt{traversedTrees}.

Metoda \texttt{preorder\_traverse\_recursive()}, podobnie jak inne prywatne metody jest zawarta na listingu nr. \ref{lst:BSTprivate}, konkretnie od linijki nr. 50 do linijki nr. 59. Na początku metody, sprawdzane jest czy parametr \texttt{node} jest równy \texttt{nullptr} - oznaczałoby to, że dotarło się do końca danej ścieżki w drzewie i należy się wrócić. Następnie bieżący \texttt{node} jest pushowany do wektora \texttt{traversedTrees}, po czym w linijkach nr. 57 i 58, wywoływana jest dwukrotnie metoda \texttt{preorder\_traverse\_recursive()}. Za pierwszym razem na lewy, węzeł, a za drugim na prawym. Takie wywołanie sprawi, że najpierw odwiedzone zostaną lewe gałęzie, a dopiero później prawe, zgodnie z porządkiem preorder.

Wracając z metody \texttt{preorder\_traverse\_recursive()} do metody \texttt{traverse\_preorder()} i listingu nr. \ref{lst:BSTtraverse_preorder}, w linijce nr. 6, widać że do \texttt{traversedTrees} aplikowana jest transformacja przy użyciu biblioteki \texttt{std::ranges}, powodująca, że zwrócony zostanie wektor zawartości poszczególnych węzłów. 

\subsection{Ciekawe fragmenty kodu}
